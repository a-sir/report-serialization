\documentclass[8pt]{beamer}

\usepackage{default}
\usepackage{pscyr} % High quality fonts
\usepackage{color}
\usepackage{minted}

\usepackage{textcomp}

\graphicspath{{images/}}

% \newcommand{\boldunderline}[1]{\textbf{\underline{#1}}}
% \newcommand{\boldit}[1]{\textbf{\textit{#1}}}

\mode<presentation>{\usetheme{Frankfurt}}

\begin{document}

\title{Java binary serialization review}

\author{Sirenko A.}

\begin{frame}
  \titlepage
  \begin{center}
    2014
  \end{center}
\end{frame}

\section{Introduction}
\renewcommand{\partname}{Serialization}
\subsection{\partname}
\begin{frame}{\partname}
  \centering{
    We use serialization in 3 cases:
    \begin{enumerate}
      \item to store;
      \item to transmit;
      \item to loose;
    \end{enumerate}
  }
  \centering{
    Main features to consider:
    \begin{enumerate}
      \item CPU usage;
      \item serialized size;
      \item freedom to change format;
      \item laboriousness;
      \item platform-dependence;
    \end{enumerate}
  }
\end{frame}

\renewcommand{\partname}{JDK tools}
\subsection{\partname}
\begin{frame}{\partname}
  \begin{itemize}
    \item Standard serialization: smallest effort / worst performance;
    \item Externalizable serialization: full control over
    serialization, define your binary format by hand;
  \end{itemize}
  \begin{table}
    \begin{tabular}{l | c | c | c | c | c }
      serialization  & speed & size & freedom & laboriousness & platform \\
      &       &      & to change &               & dependence\\
      \hline \hline
      standard       & poor & big   & average & low  & only java  \\
      \hline
      externalizable & fast & small & poor    & high & independent
    \end{tabular}

    \caption{Standard serialization vs Externalizable}
  \end{table}

\end{frame}

\section{Third party libs}
\renewcommand{\partname}{Why use something else?}
\subsection{\partname}
\begin{frame}{\partname}
\centering{
  Motivation:
  \begin{itemize}
    \item conveniently simple and fast approaches;
    \item better adopted for our data: optional params, variable-length numbers, aliases;
    \item independence from language necessary in some cases.
  \end{itemize}
}
\centering{
  Here we consider:
  \begin{itemize}
    \item standard serialization;
    \item externalizable;
    \item protobuf - \url{https://code.google.com/p/protobuf/};
    \item avro - \url{http://avro.apache.org/};
    \item kryo - \url{https://github.com/EsotericSoftware/kryo}.
  \end{itemize}
}
\end{frame}

\renewcommand{\partname}{Protobuf}
\subsection{\partname}
\begin{frame}[fragile]{\partname}
  Protobuf info
  \begin{itemize}
    \item needs proto-file with description of your data in primitives;
    \item java, python, c++ officially supported;
    \item performant and compact;
  \end{itemize}

.proto file:
\begin{minted}[fontsize=\tiny]{java}
package tutorial;

option java_package = "com.example.tutorial";
option java_outer_classname = "AddressBookProtos";

message Person {
  required string name = 1;
  required int32 id = 2;
  optional string email = 3;

  enum PhoneType {
    MOBILE = 0;
    HOME = 1;
 }

  message PhoneNumber {
    required string number = 1;
    optional PhoneType type = 2 [default = HOME];
  }

  repeated PhoneNumber phone = 4;
}

message AddressBook {
  repeated Person person = 1;
}

\end{minted}

\end{frame}

\renewcommand{\partname}{Avro}
\subsection{\partname}
\begin{frame}[fragile]{\partname}
  Avro info
  \begin{itemize}
    \item needs schema-file, can use different schemes to read and write;
    \item java, python, C\\C++, Ruby;
    \item schema stored with data $\Rightarrow$ compact files; no manual
    field-IDs;
    \item code-generational optional for static languages;
  \end{itemize}

  \begin{minted}[fontsize=\tiny]{json}
{
    "type": "record",
    "name": "Employee",
    "fields": [
        {"name": "name", "type": "string"},
        {"name": "age", "type": "int"},
        {"name": "emails", "type": {"type": "array", "items": "string"}},
        {"name": "boss", "type": ["Employee","null"]}
    ]
}
  \end{minted}

\end{frame}

\renewcommand{\partname}{Kryo}
\subsection{\partname}
\begin{frame}[fragile]{\partname}
  Kryo info
\begin{itemize}
  \item Used as standard serialization with custom serializer when you
  need it;
  \item Only java;
  \item For custom serializer there are primitives and features
  (variable-length numbers, positive optimisations);
\end{itemize}

Automatic serialization $\Rightarrow$ not much difference from
standard serialization:

  \begin{minted}[fontsize=\tiny]{java}
    Kryo kryo = new Kryo();

    Output output = new Output(new FileOutputStream("file.bin"));
    SomeClass someObject = ...
    kryo.writeObject(output, someObject);
    output.close();

    Input input = new Input(new FileInputStream("file.bin"));
    SomeClass someObject = kryo.readObject(input, SomeClass.class);
    input.close();
  \end{minted}

\end{frame}

\renewcommand{\partname}{Kryo custom}
\subsection{\partname}
\begin{frame}[fragile]{\partname}

To customize serializer, submit Serializer to Kryo object or
read/write primitives using methods of Input/Output:

 \begin{minted}[fontsize=\tiny]{java}
public class StoredUrl2Serializer extends Serializer<StoredUrl2> {

    @Override
    public void write(Kryo kryo, Output output, StoredUrl2 storedUrl2) {
        output.writeByte(0); // first version of format, placeholder for future versions and optimizations
        output.writeString(this.url.toString());
        output.writeShort(crawlResultCode);
        ...
        output.writeVarInt((int) lastFetchTime, false);
        Kryo.writeObjectOrNull(output, sourceIds, String.class);
        kryo.writeObjectOrNull(output, outLinks2, OutLink[].class);
        ...
        output.writeVarLong(contentLength, false);
    }

    @Override
    public StoredUrl2 read(Kryo kryo, Input input, Class<StoredUrl2> aClass) {
       byte format;
       if ((format = input.readByte()) != 0) {
           throw new RuntimeException("Format not supported: " + format);
       }
       Url url = new Url(input.readString(), false);
       crawlResultCode = input.readShort();
       ...
       long lastFetchTime = input.readVarInt(false);
       @Nullable String sourceIds = kryo.readObjectOrNull(input, String.class);
       @Nullable OutLink[] outLinks = kryo.readObjectOrNull(input, OutLink[].class);
       long contentLength = input.readVarLong(false);
       ...
       return new StoredUrl2(url, ..., contentLength);
    }
 \end{minted}

\end{frame}


\renewcommand{\partname}{Standard vs Kryo}
\subsection{\partname}
\begin{frame}[fragile]{\partname}

Tested on core i5-2450M + SSD

Without compression:

  \begin{table}
    \begin{tabular}{l | c | c | c }
      parameter       & standard & kryo & change \% \\
      \hline \hline
      read time, msec & 4999 & 2395   & 52\%   \\
      (1.5m objects)  &      &        & faster \\
      \hline
      write time, msec& 2283 & 975   & 67\%   \\
      (500k objects)  &      &       & faster \\
      \hline
      object size     & 332 & 264   & 20\% \\
      (bytes)         &     &       & more compact \\
      \hline
    \end{tabular}

    \caption{Standard vs Kryo in planners queue}
  \end{table}

With snappy:

  \begin{table}
    \begin{tabular}{l | c | c | c }
      parameter       & standard & kryo & change \% \\
      \hline \hline
      read time, msec       & 5306 & 2626   & 51\%   \\
      (1.5m objects)        &      &        & faster \\
      \hline
      write time, msec      & 3406 & 1250   & 63\%   \\
      (500k objects)        &      &       & faster \\
      \hline
      object size           & 110 & 91   & 18\% \\
      (bytes)               &     &      & more compact \\
      \hline
    \end{tabular}

    \caption{Standard vs Kryo in planners queue}
  \end{table}


 \begin{minted}[fontsize=\tiny]{java}

\end{minted}
\end{frame}

\renewcommand{\partname}{Standard vs. Kryo}
\subsection{\partname}
\begin{frame}[fragile]{\partname}
  \begin{itemize}
    \item<1-> \large{Useless improvement.}
    \item<2-> \large{Cause we don't care about speed :)}
    \item<3-> \large{Let's compress more:}
\newline
GZip compress well, but too slow.
Deflate with $BEST\_SPEED$ option perform between GZip and Snappy.

  \begin{minted}[fontsize=\small]{java}

    public KryoStoredUrlWriter create(@NotNull File file)
        throws IOException {

      Deflater def = new Deflater(Deflater.BEST_SPEED);

      return new KryoStoredUrlWriter(
        new DeflaterOutputStream(
          new BufferedOutputStream(new FileOutputStream(file), 4 * 1024),
          def
        )
      );
    }
  \end{minted}
  \end{itemize}

\end{frame}

\section{Results}
\renewcommand{\partname}{Standard vs. Kryo in queue}
\subsection{\partname}
\begin{frame}[fragile]{\partname}

  \begin{table}
    \begin{tabular}{l | c | c | c }
      parameter       & standard + snappy & kryo + deflate & change \% \\
      \hline \hline
      read time, msec & 5292              & 4150           & 22\%   \\
      (1.5m objects)  &                   &                & faster \\
      \hline
      write time, msec& 2529              & 2461           & same   \\
      (500k objects)  &                   &                &        \\
      \hline
      object size     & 110               & 73             & 33\% \\
      (bytes)         &                   &                & more compact \\
      \hline
      reconstruction time  & 240          & 210            & 12.5\% \\
      planner2v (min)      &              &                & faster \\
      \hline
    \end{tabular}
    \caption{Standard + snappy vs Kryo + deflate}
  \end{table}

  Space economy on planner2v (5 storages + 3 queue backups): 250Gb

\end{frame}

\renewcommand{\partname}{Choice}
\subsection{\partname}
\begin{frame}[fragile]{\partname}

My subjective opinion to choose serialization:
\begin{itemize}
  \item<1-> Use standard serialization if {
    \begin{itemize}
      \item<1-> Someone will kill you if you write code slowly
      \item<2-> You are making task on job interview in Big Bank
      \item<3-> Serialization don't affect performance of system
      \item<4-> You add/remove fields sometimes, but lazy to update serializer.
    \end{itemize}
  }
  \item<5-> Use Kryo if {
    \begin{itemize}
      \item<6-> Someone will kill you if you write code slowly;
      \item<7-> Data is processed by java only;
      \item<8-> Twice speed of standard serialization is enough;
      \item<9-> Subzero is your favorite fighter in Mortal Combat.
    \end{itemize}
  }


  \item<10-> Use Avro if {
    \begin{itemize}
      \item<11-> Data is processed in different languages;
      \item<12-> You want to be unique:)
    \end{itemize}
  }

  \item<13-> Use Protobuf if {
    \begin{itemize}
      \item<14-> you are working at Google;
      \item<15-> Data is processed in different languages;
      \item<16-> Need most performant/compact result;
    \end{itemize}
  }

  \item<17-> Use Externalizable if {
    \begin{itemize}
      \item<18-> you are unique - want implement all the features by yourself;
      \item<19-> already working in Big Bank;
      \item<20-> watched only first 3 slides of this presentation;
    \end{itemize}
  }


\end{itemize}
\end{frame}

\end{document}
